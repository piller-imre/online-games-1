\Chapter{Implementáció}

%(6-8 oldal)

% TODO: Ide jöhetnek maguk a kódok
% Meg kell indokolni, hogy miért pont a felhasznált technológiákra esett a választás
% JWT használata
% Ebbe a fejezetbe kerülhetnek majd a kódpéldák. Az elejébe annyira nem kellene majd.
% A játékállapot kezelésének konkrét implementációja



\SubSection{UserController}
A UserController fő feladata magával a felhasználó személyével kapcsolatos kérések kezelése. Ide tartozik
\begin{itemize}
	\item a regisztráció,
	\item a bejelentkezés,
	\item a kijelentkezés,
	\item jelszó megváltoztatása,
	\item a elfelejtett jelszó kezelése,
	\item és a felhasználó törlése.
\end{itemize}




\SubSection{MatchController}
A MatchController, ahogy a neve is sugallja a meccsek kezelésével kapcsolatos kéréseket dolgozza fel. Ennek feladata
\begin{itemize}
	\item a kihívások (meccs kezdeményezések) létrehozása,
	\item törlése,
	\item a kihívások listájának lekérdezése,
	\item és az aktív játékmenet létrehozása elfogadáskor.
\end{itemize}




\SubSection{FiveInARowController}
Az amőba kontrollere értelem szerűen már a megkezdett, amőba játéktípusú meccs kéréseit kezeli. Ezek az alábbiak:
\begin{itemize}
	\item az aktív játékos lép,
	\item lépett -e az ellenfél,
	\item lejárt a lépéshez megengedett idő (timeout),
	\item játék feladása
\end{itemize}




\SubSection{CheckersController}
Hasonló jellegű kéréseket kell teljesítenie, mint egyéb típusú játékok esetén, a játék sajátosságai miatt minden játéktípusnak külön kontrollerre és metódusokra van szüksége.
